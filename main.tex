\documentclass[12pt,a4paper]{report}
\usepackage[margin=1in]{geometry}
 \geometry{a4paper, right=25mm, rmargin=0mm, outer=25mm, left=0mm, lmargin=0mm,  inner=25mm,
 top=0mm, tmargin=25mm, bottom=0mm, bmargin=25mm
}

%Packages
%\usepackage{sectsty}  % Gera warnings com o LuaTex
\usepackage[portuguese, english]{babel}
\usepackage{graphicx}
\usepackage{blindtext}
\usepackage{fontspec}
\usepackage{multirow}
\usepackage{afterpage}
\usepackage{layout}
\usepackage{blindtext}
\usepackage{fancyhdr}
\usepackage{xcolor}
\usepackage{color}
\usepackage{apacite}
\usepackage{caption}
\usepackage{fix-cm}
\usepackage{float}
\usepackage{makecell}
\usepackage{tabularx}
\usepackage{titlesec}
\usepackage{etoolbox}
\usepackage[acronym]{glossaries}
\usepackage{pdfpages}
\usepackage{longtable}

%Definições dos Títulos
\titleformat{\chapter}[display]
  {\normalfont\bfseries\color{darkgray}}{}{0pt}{\Large}
\titleformat*{\section}{\color{darkgray}\Large\bfseries}
\titleformat*{\subsection}{\color{darkgray}\Large\bfseries\color{darkgray}}
\titleformat*{\subsubsection}{\color{darkgray}\Large\bfseries}
\setcounter{secnumdepth}{3}
\renewcommand{\thesubsubsection}{\alph{subsubsection})}

%Definições do Documento
\renewcommand{\baselinestretch}{1.5} 
\setmainfont{NewsGotT.ttf}
\setmainfont[
%Mapping=tex-text,  %Gera warning com LuaTex
Ligatures=TeX,
AutoFakeSlant=0.20,
BoldFont=NewsGotTBold.ttf
]{NewsGotT.ttf} 

\titlespacing{\chapter}{0pt}{-40pt}{10pt} 


%Definições do Cabeçalho
\fancyhf{}
\renewcommand{\chaptermark}[1]{\markboth{#1}{}}
\renewcommand{\sectionmark}[1]{\markright{#1}{}}
\fancyhead[R]{\textcolor{gray}{Análise Crítica de Modelos Simples para a Abordagem à Cibersegurança e Privacidade }} 

%Definições das Secções
\fancyhead[L]{}
\setlength{\headheight}{15pt}
\pagestyle{fancy}
\patchcmd{\chapter}{\thispagestyle{plain}}{\thispagestyle{fancy}}{}{}
\patchcmd{\headrule}{\hrule}{\color{black}\hrule}{}{} %Change color Hrule


\begin{document}
\fancyfoot[R]{\thepage}
\pagenumbering{roman}

%Capa 
\begin{titlepage}
    \includepdf[pages={1}]{chapter/pdf/capaTP3.pdf}
\end{titlepage}


%Índice
\renewcommand{\contentsname}{Índice}
\tableofcontents
\clearpage
\let\oldnumberline\numberline% Copy \numberline into \oldnumberline
\renewcommand{\numberline}[1]{\hspace*{-1.5em}}% Remove number argument


%Definições do Rodapé
\fancyfoot[R]{\thepage}

%Enquadramento
\setcounter{chapter}{1}
\chapter*{\thechapter \quad Enquadramento e Motivação}
\addcontentsline{toc}{chapter}{\thechapter \quad Enquadramento e Motivação}
\pagenumbering{arabic}
\paragraph{}
Lorem ipsum dolor sit amet, consectetur adipiscing elit. Donec eget dictum erat. Vestibulum ante ipsum primis in faucibus orci luctus et ultrices posuere cubilia curae; Nunc placerat turpis magna, eu tempus risus finibus vitae. Donec at nunc posuere, porta sem et, congue erat. Aenean suscipit faucibus scelerisque. Aenean in odio elementum, blandit tellus et, convallis nisl. Suspendisse ut nisi at augue tristique tempus. Cras lobortis arcu in felis pulvinar convallis. Aliquam sollicitudin neque non elementum iaculis. Pellentesque mattis sapien a justo pellentesque, ut ornare mauris accumsan. Nulla facilisi. Maecenas posuere luctus ipsum, a faucibus enim lacinia et. Nulla id vehicula urna, vitae feugiat arcu. Nulla laoreet libero quam, quis varius odio egestas vel. Fusce tempus orci nec ligula aliquam, aliquet aliquam ex dictum. Nullam pharetra tincidunt tellus, quis facilisis nisl luctus sit amet.

Phasellus gravida mi nec tristique pharetra. Mauris feugiat condimentum convallis. Sed hendrerit erat at dui congue, a ultrices nulla tincidunt. Aenean sed dolor interdum, laoreet lectus in, placerat libero. Mauris elementum gravida sem, sit amet posuere erat fringilla eu. Mauris sit amet ex interdum orci aliquam efficitur et id mi. Nulla posuere nec odio sed cursus. Donec dictum ex vel magna maximus faucibus. Suspendisse in tincidunt odio. Donec tellus risus, rhoncus quis fermentum id, tincidunt id lorem. Quisque tincidunt luctus massa, vel convallis arcu tincidunt vitae. Proin dignissim purus eget cursus aliquam.
 %Added from chapter directory

%\newpage\null\thispagestyle{empty}\newpage

%Objetivos e Resultados Esperados
\setcounter{chapter}{2}
\setcounter{section}{0}
\chapter*{\thechapter \quad Objetivos e Resultados Esperados}
\addcontentsline{toc}{chapter}{\thechapter \quad Objetivos e Resultados Esperados}
\paragraph{}

Os principais objetivos são:
\begin{itemize}
   \item Identificar ;
   \item Estabelecer ;
   \item Propor .
\end{itemize}

Com a realização deste projeto esperam-se os seguintes resultados:
\begin{itemize}
   \item Análise GAP 
\end{itemize}



%Estado da Arte
\setcounter{chapter}{3}
\setcounter{section}{0}
\chapter*{\thechapter \quad Estado da Arte}
\addcontentsline{toc}{chapter}{\thechapter \quad Estado da Arte}
\paragraph{}
Este trabalho foi realizado utilizando a plataforma online chamada Overleaf onde tivemos de criar um template base para este Relatório do TP3 e TP4 de LaTeX e Github. 
Tivemos sempre com um intuito de descobrir novos comandos de importação e de estilização de relatório, de forma a mais tarde podermos reutilizar.






%Caso de Estudo
\setcounter{chapter}{4}
\setcounter{section}{0}
\chapter*{\thechapter \quad Caso de Estudo}
\addcontentsline{toc}{chapter}{\thechapter \quad Caso de Estudo}
\paragraph{}
Algum texto



%Bibliografia
\renewcommand\bibname{Referências Bibliográficas}
\bibliographystyle{apacite}
\bibliography{mybib.bib}

%Anexos
\addcontentsline{toc}{chapter}{Calendarização}

\includepdf[pages=-,angle=0]{chapter/pdf/Plano_de_Atividades.pdf}

\end{document}  